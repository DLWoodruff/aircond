\documentclass{article}
\usepackage{amsmath,amssymb}
\usepackage[margin=1in]{geometry}
\usepackage{hyperref}

\title{aircondMulti: A Multi-Product Extension of the aircond Model}
\author{David L.\ Woodruff\\
Graduate School of Management, UC Davis\\
\texttt{DLWoodruff@UCDavis.edu}}
\date{February 2026}

\begin{document}
\maketitle

\begin{abstract}
  We describe \texttt{aircondMulti}, a multi-product extension of
  the \texttt{aircond} model documented in~\cite{aircond}.  The
  extension introduces $P$ products that share a single regular-time
  production capacity while maintaining independent demand streams,
  inventories, and overtime production.  The model is implemented as
  an mpi-sppy scenario creator and supports bundling.
  See \url{https://github.com/DLWoodruff/aircond}.
\end{abstract}

\section{Overview}

The single-product \texttt{aircond} model~\cite{aircond} is a
multi-stage stochastic production planning problem with regular-time
and overtime production, inventory carry-forward, and optional
start-up costs.  The \texttt{aircondMulti} extension generalises
this to $P$ products ($P \ge 1$) that \emph{share} a common
regular-time capacity constraint while each product has its own
demand process, inventory state, and overtime production.

The number of products and the degree of cost differentiation are
controlled by two new parameters: \texttt{num\_products} ($P$,
default~2) and \texttt{cost\_spread} ($\alpha$, default~0.1).
Product~$p$ (indexed from~0) pays costs that are a factor of
$(1 + p\alpha)$ times the base regular-time and overtime costs.

\section{Variables}

We retain the notation of~\cite{aircond} with the addition of a
product index $p \in \{0, \dots, P-1\}$.  At each stage $t$:
\begin{align*}
  x_{t,p} &\in \mathbb{R}_{\ge 0} & &\text{regular-time production of product $p$}\\
  w_{t,p} &\in \mathbb{R}_{\ge 0} & &\text{overtime production of product $p$}\\
  y_{t,p} &\in \mathbb{R}          & &\text{inventory of product $p$ (positive or negative)}\\
  y^+_{t,p} &\in \mathbb{R}_{\ge 0}& &\text{positive part of $y_{t,p}$}\\
  y^-_{t,p} &\in \mathbb{R}_{\ge 0}& &\text{negative part of $|y_{t,p}|$}\\
  \delta_t &\in \{0,1\}            & &\text{start-up indicator (shared across products)}
\end{align*}

\section{Parameters}

All parameters from~\cite{aircond} are retained with the following
modifications:
\begin{itemize}
  \item $K_t$ (regular-time capacity): now shared across all products,
    i.e.\ $\sum_{p} x_{t,p} \le K_t$.
  \item $C_t$ and $O_t$ (base regular and overtime costs): product~$p$
    pays $C_t(1 + p\alpha)$ and $O_t(1 + p\alpha)$, respectively.
  \item $y_0$ (beginning inventory) and $D_1$ (starting demand) are
    each divided equally across $P$ products.
  \item $D_{t,p}$ (demand): each product has an independent demand
    stream; see Section~\ref{sec:demand}.
\end{itemize}

Two new parameters are introduced:
\begin{itemize}
  \item \texttt{num\_products} ($P$): the number of products (default~2).
  \item \texttt{cost\_spread} ($\alpha$): the per-product cost increment
    factor (default~0.1).
\end{itemize}

\section{Objective and Constraints}

The stage-$t$ cost for product~$p$ is
\[
  C_t(1+p\alpha)\, x_{t,p}
  + O_t(1+p\alpha)\, w_{t,p}
  + \begin{cases}
      I_t\, y^+_{t,p} + N_t\, y^-_{t,p} + Q_t (y^-_{t,p})^2, & t < T\\
      N_T^{\mathrm{last}}\, y^+_{T,p} + N_T\, y^-_{T,p}, & t = T
    \end{cases}
\]
where $N_T^{\mathrm{last}} < 0$ is a salvage value for leftover
inventory in the final stage.

The total stage-$t$ cost sums over products, plus an optional shared
start-up cost:
\[
  \phi_t = \sum_{p=0}^{P-1}
    \bigl[C_t(1+p\alpha)\, x_{t,p} + O_t(1+p\alpha)\, w_{t,p}
          + \text{inventory cost}_{t,p}\bigr]
    + S_t\, \delta_t.
\]

\paragraph{Constraints.}
For each stage~$t$ and product~$p$:
\begin{align}
  \sum_{p=0}^{P-1} x_{t,p} &\le K_t
    \label{eq:cap}\\
  y_{t-1,p} + x_{t,p} + w_{t,p} - y_{t,p} &= D_{t,p}
    \label{eq:balance}\\
  y^+_{t,p} - y^-_{t,p} &= y_{t,p}
    \label{eq:invsplit}\\
  M\,\delta_t &\ge \sum_{p=0}^{P-1}(x_{t,p} + w_{t,p})
    \label{eq:startup}
\end{align}
Constraint~\eqref{eq:cap} is the shared regular-time capacity.
Constraint~\eqref{eq:balance} is the per-product material balance
($y_{0,p} = y_0 / P$).  Constraint~\eqref{eq:startup} links the
shared start-up binary to total production across all products.

\section{Demand Generation}
\label{sec:demand}

Each product has an independent demand stream generated by the same
Gaussian random walk as in~\cite{aircond}, but with a per-product
seed offset.  The demand for product~$p$ at tree node with index
$\nu$ is generated using seed
\[
  \text{seed} = s_0 + p \cdot 100{,}000 + b \cdot 100 + \nu
\]
where $s_0$ is the base \texttt{start\_seed}, $b$ is the bundle
index (zero for non-bundled problems), and $\nu$ is the tree-node
index computed by \texttt{sputils.node\_idx()}.

The large product offset (100\,000) ensures that demand streams for
different products never collide.  The bundle offset (100) ensures
that different bundles see distinct demand realisations even though
mpi-sppy's bundling mechanism passes identical sub-tree branching
factors to every bundle (see Section~\ref{sec:bundling}).

Starting demand for each product is $D_1 / P$ (deterministic at the
root node).

\section{Bundling Support}
\label{sec:bundling}

When mpi-sppy creates bundles for a tree with branching factors
$[B_1, B_2, \dots, B_T]$ and $s$ scenarios per bundle, it passes
sub-tree branching factors $[1, B_2, \dots, B_T]$ to the scenario
creator for each bundle.  Because $B_1' = 1$ in the sub-tree, all
bundles would otherwise share identical node indices and therefore
identical demand realisations.

To break this degeneracy, the scenario creator computes
\[
  b = \bigl\lfloor \texttt{scennum} \,/\, \textstyle\prod_i B_i' \bigr\rfloor
\]
and adds $b \cdot 100$ to the demand seed.  For the non-bundled case,
all scenario numbers satisfy $\texttt{scennum} < \prod B_i$, so
$b = 0$ and behaviour is unchanged.

\section{mpi-sppy Interface}

The module provides the standard mpi-sppy scenario-creator interface:
\begin{itemize}
  \item \texttt{scenario\_creator(sname, **kwargs)}: builds a Pyomo
    \texttt{ConcreteModel} for the given scenario name.
  \item \texttt{scenario\_names\_creator(num\_scens, start=None)}:
    generates scenario name strings.
  \item \texttt{inparser\_adder(cfg)}: registers all command-line
    arguments.
  \item \texttt{kw\_creator(cfg)}: extracts keyword arguments from
    the configuration.
\end{itemize}

Nonant variables are \texttt{RegularProd[p]} and
\texttt{OvertimeProd[p]} for all products at all stages except the
last.  Supplementary EF variables include \texttt{Inventory[p]} for
each product and the optional \texttt{StartUp} binary.

\section{Default Parameter Values}

The module defines the following defaults (matching the
\texttt{parms} dictionary in the source code):

\medskip
\begin{center}
\begin{tabular}{lll}
  \hline
  Parameter & Default & Description \\
  \hline
  \texttt{num\_products}    & 2     & number of products \\
  \texttt{cost\_spread}     & 0.1   & cost differentiation factor $\alpha$ \\
  \texttt{starting\_d}      & 200   & total starting demand (split across products) \\
  \texttt{BeginInventory}   & 200   & total initial inventory (split across products) \\
  \texttt{Capacity}         & 200   & shared regular-time capacity $K$ \\
  \texttt{RegularProdCost}  & 1.0   & base regular-time cost $C$ \\
  \texttt{OvertimeProdCost} & 3.0   & base overtime cost $O$ \\
  \texttt{InventoryCost}    & 0.5   & per-unit holding cost $I$ \\
  \texttt{NegInventoryCost} & 5.0   & per-unit backorder cost $N$ \\
  \texttt{LastInventoryCost}& $-0.8$& final-stage salvage value \\
  \texttt{sigma\_dev}       & 40    & demand std.\ deviation $\sigma$ \\
  \texttt{mu\_dev}          & 0     & demand mean deviation $\mu$ \\
  \texttt{start\_ups}       & False & include start-up costs (MIP) \\
  \texttt{StartUpCost}      & 300   & start-up cost $S$ \\
  \texttt{start\_seed}      & 1134  & base random seed $s_0$ \\
  \texttt{QuadShortCoeff}   & 0     & quadratic backorder coefficient $Q$ \\
  \hline
\end{tabular}
\end{center}

\begin{thebibliography}{9}
\bibitem{aircond}
  D.~L.\ Woodruff, B.~C.\ Knight, X.\ Chen, and S.\ Cazaux.
  \newblock aircond: An Example for Optimization Under Uncertainty.
  \newblock \url{https://github.com/DLWoodruff/aircond}, 2022.

\bibitem{mpisppy}
  B.~Knueven, D.~Mildebrath, C.~Muir, J.~P.\ Watson,
  and D.~L.\ Woodruff.
  \newblock A Parallel Hub-and-Spoke System for Large-Scale Scenario-Based
  Optimization Under Uncertainty.
  \newblock \emph{Mathematical Programming Computation}, 15:591--619, 2023.
\end{thebibliography}

\end{document}
